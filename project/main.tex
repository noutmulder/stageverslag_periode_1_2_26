\documentclass[a4paper, 12pt]{article} 

% ---------- Packages ----------
\usepackage[english]{babel}             
\usepackage[utf8]{inputenc}             
\usepackage[T1]{fontenc}                
\usepackage{geometry}                   
\usepackage{graphicx}                   
\usepackage{amsmath, amssymb}           
\usepackage[numbers]{natbib}            
\usepackage{hyperref}                   
\hypersetup{pdfborder = 0 0 0}          
\usepackage{fancyhdr}                   
\usepackage{setspace}                   
\usepackage{enumitem}                   
\usepackage{listings}                   
\usepackage{xcolor}                     
\usepackage{float}                      
\usepackage{titlesec}                   
\usepackage{caption}                    
\usepackage{tocbibind}                  
\usepackage{url}                        
\usepackage{subcaption}                 
\usepackage{titling}                    
\usepackage{multicol}                   

\setlength{\parindent}{0pt}            

\geometry{ 
  a4paper,
  left=2cm,
  right=2cm,
  top=2cm,
  bottom=2cm
}

% ---------- Header / Footer ----------
\pagestyle{fancy}
\fancyhf{}
\fancyhead[L]{\footnotesize\itshape Internship Report DeVi Comfort \textbullet{} 2026}
\fancyhead[R]{\footnotesize\thepage}
\renewcommand{\headrulewidth}{0.4pt}
\renewcommand{\footrulewidth}{0pt}

% ---------- Title ----------
\title{Internship Research Report Period 1 and 2}
\author{Nout Mulder}
\date{2026}

% ---------- Document ----------
\begin{document}

% ---------- Journal-style Title Block ----------
\thispagestyle{fancy}
\vspace*{0.5cm}
{\small\itshape Internship Report DeVi Comfort. 2026; 1(1): 1--\pageref{LastPage}\par}
\vspace{0.2cm}
\hrule
\vspace{0.6cm}
\begin{center}
{\Large\bfseries INTERNSHIP RESEARCH REPORT PERIOD 1 AND 2\par}
\vspace{0.3cm}
{\normalsize Nout Mulder\par}
\vspace{0.2cm}
{\small DeVi Comfort \textbullet{} Engineering \textbullet{} Opmeer\par}
\end{center}
\vspace{0.4cm}
\hrule
\vspace{0.6cm}

\section*{SUMMARY}
\setlength{\parskip}{6pt}
This research focuses on developing a 3D visualization (``virtual ghost'') to make
faults and deviations in position and movement of a stairlift faster and more
intuitive to interpret within DeVi Comfort. The assignment is carried out
iteratively with prototyping and evaluation based on practical scenarios.
\vspace{0.4cm}
\hrule
\vspace{0.6cm}

\setlength{\columnsep}{0.8cm}
\begin{multicols}{2}

% ============================================================
\section{Introduction}

\subsection{Context}
DeVi Comfort designs and produces stairlift systems and develops the accompanying
software in-house. The company continually works on reliability, safety and
maintainability of the product. During service and further development it is
important that faults can be recognized and analyzed quickly.

\subsection{Motivation}
Textual fault information and error codes are correct, but in practice it takes
time to interpret them properly. When the cause of a fault is not quickly clear,
this leads to additional diagnosis time, longer downtime and higher support and
service costs. A clear visual representation of position, movement and deviations
can reduce this.

\subsection{Research Goal}
The goal is to develop a 3D visualization (``virtual ghost'') that shows the
stairlift on the rail and provides insight into movement and status. This should
make faults and warnings faster and more intuitive to interpret, so that diagnosis
and recovery are more efficient.

\subsection{Reading Guide}
Chapter 2 describes the problem statement and research questions. Chapter 3 provides
the theoretical framework. Chapter 4 describes the method. Chapter 5 presents the
results. Chapter 6 concludes with the conclusion and recommendations.

% ============================================================
\section{Problem Statement}

\subsection{Problem Description}
Faults and warnings of the stairlift are not always immediately understandable
based only on error codes and status bits. As a result, it is difficult to quickly
determine whether the stairlift is moving correctly, has lost its position, or is
in a fault condition. This slows down diagnosis and recovery.

\subsection{Main Question}
Which data and design choices are needed to develop a 3D ``virtual ghost'' visualization
that makes faults and deviations in position and movement of a stairlift understandable
for fault analysis within DeVi Comfort?

\subsection{Subquestions}
\begin{itemize}[leftmargin=*]
  \item Which faults and warnings are most relevant to visualize?
  \item Which technical inputs (position, speed, status, error codes) are needed for a reliable and stable 3D representation?
  \item How is the expected position determined and how is the difference with the measured position calculated and interpreted?
  \item Which visualization forms make deviations between expected and measured position most understandable for engineers and service?
  \item How can the prototype be validated with practical scenarios (e.g., impact on interpretation errors and diagnosis time)?
\end{itemize}

% ============================================================
\section{Theoretical Framework}

\subsection{Background Information}
DeVi Comfort works with complete stairlift systems in which mechanics, electronics
and software come together. For maintenance and service, insight into the current
and expected position of the lift is essential.

\subsection{Relevant Theory}
Relevant themes include system diagnostics, visualization of technical status
information and interpreting deviations between measured and expected values in
mechatronic systems.

\subsection{Existing Solutions}
Within DeVi Comfort, fault information is currently offered primarily as text.
This research explores how a 3D visualization can complement this for faster
interpretation.

% ============================================================
\section{Method}

\subsection{Research Approach}
The assignment is carried out iteratively with prototyping and evaluation. The cycle
consists of analysis, design, implementation and validation based on practical
scenarios and measurable results.

\subsection{Research Methods}
Methods include document analysis, requirements analysis, visualization design,
prototyping and evaluation with involved engineers and service.

\subsection{Tools and Techniques}
Work is done with a 3D visualization environment (``virtual ghost'') and data from
stairlift controllers such as position, speed, status and error codes.

% ============================================================
\section{Results}

\subsection{Findings}
To be completed based on analysis and prototyping.

\subsection{Analysis}
To be completed based on evaluations and measurements.

\subsection{Results Overview}
To be completed based on the final results.

% ============================================================
\section{Conclusion}

\subsection{Answer to Main Question}
To be completed after the research is finished.

\subsection{Key Insights}
To be completed after the research is finished.

\subsection{Recommendations}
To be completed after the research is finished.

% ============================================================
\end{multicols}

\newpage
\bibliographystyle{plainnat}
\bibliography{bibliography}

\label{LastPage}

% ============================================================

\end{document}
